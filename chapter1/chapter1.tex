
\chapter{Introduction}

Welcome to your thesis template! To begin with, a brief explanation on how to use the template. The main \TeX\ (pronounced as \textit{tech}) file is located in \texttt{main/main.tex} and will be the main file to compile. Individual chapters also compile independently (located in e.g., \texttt{chapter1/chapter1.tex}) which is useful when compilation times exceed the minute. This independent compilation is a good tool, since you'll be adding a ton of text it is a pain to always wait for the \texttt{main.tex} file to finish compiling.

If you are using Linux or MacOS a few commands come in handy: \texttt{pdflatex} and \texttt{\texttt{bibtex}}. If you're using VS Code or Atom, or similar text editor they will automatically call these commands under-the-hood in order to compile your manuscript. For details on how to setup writing and coding environments check the \texttt{README.md} file in the GitHub repository.

\LaTeX\ is not a word processing system, meaning that all information will be first compiled to then be seen in the PDF document. Plain text is low weight, very fast, and does not need additional resources for computers now-a-days to use and store. On the contrary word processors, are \textit{laggy}, they tend to be very slow and/or all of a sudden stop working. \LaTeX\ gives you the advantage of just focusing on the content and then on the formatting, which is in fact the important thing.

Over the past decade online \LaTeX\ services have been popularized a lot. Services such as Overleaf or Sharelatex are great to collaborate, but lose the main principles of \LaTeX. Control of your environment, usage of bash and other coding utilities, customization and cross-reference for multiple files make the usage of Overleaf for large documents such as a thesis very tedious. In addition, the option to work off-line (distraction free) in a simple text editor is not comparable.
%
Other drawbacks of Overleaf include limited access to custom packages and system-level tools, slower compilation times for large projects, and potential privacy concerns since your files are stored on third-party servers. Overleaf also restricts advanced automation and scripting, making it harder to integrate with version control systems like \texttt{git} or to automate builds and testing (bash and Python scripting). Finally, dependency on internet connectivity can be problematic, and the platform may not support all features or workflows required for complex scientific documents.

Open source software has played a crucial role in the development and widespread adoption of \LaTeX. The \LaTeX\ system itself is open source, allowing users to freely modify, distribute, and improve the codebase. This collaborative environment has led to a rich ecosystem of packages, templates, and tools contributed by the global community. Open source principles ensure that \LaTeX\ remains accessible, extensible, and adaptable for a wide range of scientific, academic, and technical writing needs. It is free of use and should not be restricted by a paywall to access all tis features.

There are multiple \LaTeX\ tutorials out there so I do not need to convince you \LaTeX\ is great, and the intention of this template is just to show you those tools that I find the most useful to write your thesis. 

One final word about customization. Here I hand to you the full customization of your document, meaning that the possibilities are infinite. Using fixed and not flexible templates will obscure your way of learning how to code here, and may not be as useful for a potential academic career where debugging \LaTeX\ code is a must.

Errors and warnings are normal, but it is strongly advice to minimize them as much as you can. The idea of this code is to be as portable as possible, light weight and easy to compile. If you have problems read online, and/or ask for help!

\section{Good practices while writing in \LaTeX}

\TeX and \LaTeX typography are key for a good document and are the main reason why we bother in writing papers in a \LaTeX\ rather than in a rich text environment.

\begin{enumerate}

  \item Proper use of math-mode and text-mode. When writing equations and variables within the text you must always keep those variables in the same notation. E.g.,
  \begin{equation}
      E = mc^2.
  \end{equation}
  Then when referencing the variables in the text you must include them as they were initially defined, viz., $E$ (\verb|$E$|) and not E. In addition, it is a good practice to keep variables different from one another as much as you can, energy $E$ and electric field $E$, perhaps use a slightly different variable for them. 

  \item \textit{Use of the cross-references}, which comes by default, within \TeX. This is very important to not fall in redundant/repeated information, unknown location of things, and simplicity of the text. With this I mean the use of \verb|\label{...}| and \verb|\ref{...}|. Every time you define a figure, always use it (otherwise what's the point of the figure?). In addition, there is non explicit phrases such as: ``in the fig below'', ``the previous equation'', ``the previous section'', etc. All those should be immediately corrected with a proper cross-reference otherwise may lead to confusion (besides the fact that the \TeX-compiler automatically adjust the text and figures to optimize the space). There are very useful packages that make this process even easier (than the native \TeX), you may want to check \verb|\usepackage{cleveref}|\footnote{\url{https://ctan.org/pkg/cleveref}.}, for example.
  
  \item Acronyms may lead to a lot of unorganized text if not used properly. In general, acronyms should be defined once, or ``once in a while'' recall them within the text (e.g., within every chapter in a long thesis). To keep track where you have defined it may lead to many edits. I suggest using \verb|\usepackage{glossaries}|\footnote{\url{https://ctan.org/pkg/glossaries}.}, it will automatically call the acronym given a predefined command, won't repeat the definition (unless you force it to do it), and also prints a full list of the used acronyms, which is super convenient. The learning curve for this may be steep, but worth it.
  
  \item The proper use of hyphens \verb|-| and it prints -, n-dash \verb|--| and it prints --, and m-dash \verb|---| and it prints ---. All of them are different! Further, there are some cases where hyphens (\verb|-|) are different from minus sign \verb|$-$| at it prints $-$. All of these have a different meaning in scientific writing. For more check: \url{https://sites.duke.edu/scientificwriting/dash-v-hyphen/}.
  
  \item Writing equations. \LaTeX\ has been built to write equations, and make them flow in the text as if they were words. This is the main philosophy when writing. In order to accomplish so, be minded of the spaces that you insert between paragraphs and equations. E.g.,
  
  \begin{verbnobox}[\small]
    Ampère's law (including the Maxwell correction) relates the generation 
    of a magnetic field curl and a time-varying electric field given a 
    source of current,
    \begin{equation}
      \nabla \times \boldsymbol{\mathcal{B}} = 
      \mu_0 \boldsymbol{\mathcal{J}} + 
      \mu_0 \epsilon_0 \frac{\partial\boldsymbol{\mathcal{E}}}{\partial t},
    \end{equation}
    where $\boldsymbol{\mathcal{B}}$, $\boldsymbol{\mathcal{E}}$, 
    and $\boldsymbol{\mathcal{J}}$ are time- and position-dependent 
    magnetic, electric, and current density fields. 
    The second term on the right in 
    \cref{eq:ampere-maxwell} corresponds to the so-called 
    ``displacement current.'' It is not a current per se, but its name 
    has stuck for historical reasons.
  \end{verbnobox}
  
  This text should compile as:

  Ampère's law (including the Maxwell correction) relates the generation of a magnetic field curl and a time-varying electric field given a source of current,
  \begin{equation}
    \nabla \times \boldsymbol{\mathcal{B}} = \mu_0 \boldsymbol{\mathcal{J}} + \mu_0 \epsilon_0 \frac{\partial \boldsymbol{\mathcal{E}}}{\partial t}, \label{eq:ampere-maxwell}
  \end{equation}
  where $\boldsymbol{\mathcal{B}}$, $\boldsymbol{\mathcal{E}}$, and $\boldsymbol{\mathcal{J}}$ are time- and position-dependent magnetic, electric, and current density fields. The second term on the right in \cref{eq:ampere-maxwell} corresponds to the so-called ``displacement current.'' It is not a current per se, but its name has stuck for historical reasons.

  In this example, please notice that text and the equation environment are packed together. There are no line breaks between them in the plain text above. If there were then \LaTeX\ will introduce a line break, removing the flow of text. Having commas and dots within the equation environment is also desired. Equations are part of the text and they should be treated as such.
  
  \item Displaying units. Along the text there will be many mentions of units, as it should be. Usually they should be expressed with a number and a fixed space, where people tend to do something like: \verb|$4\mathrm{m}$|, or \verb|$4$ m| and it prints $4$ m (notice that the number four is in math mode and not text mode!). An even better practice is to use a package to deal with this sort of things, e.g., \verb|\usepackage{siunitx}|. This package gives you the freedom of using ``quantities'' in text and math mode. For example: \verb|\qty{4}{\meter}| will \qty{4}{\m}. The cool thing about this is that you can write easily
  \begin{verbatim}
      \begin{equation}
          x^2 + y^2 = \qty{4}{\meter\squared}
      \end{equation}
  \end{verbatim}

  This will then be compiled as:
  \begin{equation}
    x^2 + y^2 = \qty{4}{\meter\squared}
  \end{equation}
  This could highly improve the quality of your science. You can also use this for Kelvins: \verb|\qty{2e4}{\kelvin}| which prints: \qty{2e4}{\kelvin}. Similarly the package offers number notations, \verb|\num{e4}| which is \num{e4} and other convenient shortcuts to list numbers and ranges for quantities and numbers. 


  \item As a general rule within your manuscript and setup. Please do not use spaces when naming files. Spaces will introduce bugs in your workflow, they won't sync nicely with \texttt{git} and will make the order of things messy. This is a rule of life than anything else!
  
  \item Spaces in \LaTeX. In general you do not need to add any sort of space between paragraphs, figures, or tables. Everything should be simply solved by including a simply plan text line break. Using commands such as \verb|\\| may be even considered harmful since it will mess up the \LaTeX\ compiler optimization. 
  
  % \item The list of figures (LoF) and list of tables (LoT) is printed with the extended version of the caption. This should not be the case. There should be a one sentence caption (only visible in LoF and LoT), and the extended version, only visible in the main text. To accomplish so use the \verb|\usepackage{caption}| package and in the captions use something like this: \verb|\caption[short caption]{long caption}|.
  
  % \item It would be best to have a header and footer in the document.
\end{enumerate}


\section{Figures}

Including figures in your thesis is one of the most important tasks. Figures are first class citizens, they should be clearly made, with a good resolution, and capable of summarizing all the information. Figures (similarly as tables) must always be called within the text with a cross-reference, and if they are not then there's no actual need for the figure.

When including images, pictures or screenshots it is advised to use as much as possible of your \verb|\textwidth| in order to fill in the entire width of the page. One common mistake that I've seen over and over the years is when people ``force'' \LaTeX\ to insert a figure in a certain place. \LaTeX\ is not a word processor and it is optimized to arrange text and figures within the compiled text. As a genera rule, write first the content and include figures, and at the last stages of the development is when you should ``fix'' figures in a certain place but not earlier. Most of the time this procedure is not even needed!
In general, you'd want to include a figure with this structure:
\begin{verbatim}
  \begin{figure}[t]
    \includegraphics{uchile}
    \caption{This is the Universidad de Chile logo.}
    \label{fig:uchile}
  \end{figure}
\end{verbatim}
Then we simply call the figure in the text as \verb|\ref{fig:uchile}| (or better using \texttt{cleveref}). Notice that I used \texttt{[t]} which will place the figure at the top of the page, again considered as a good practice.

Figure captions should be self-explanatory (and always present). They should contain all information to express the figure completely, with no need to go to the main text to understand the idea of it. Therefore, you must explain all variables and details (there is no problem with long captions). It's generally understood as a good practice to be repetitive in this case, after all most people will go quickly over the figures rather reading the entire manuscript! The sad reality.

Within the list of figures (lof) and list of tables (lot), captions will be printed completed as default in this section of the manuscript. To avoid such behavior we use the \texttt{caption} package that lets us use the short caption version: \verb|\caption[short caption]{long caption}|.

\subsection{Python figures}

To make figures using Python it is recommended to use the style file \texttt{thesis\_sty.mplstyle}, so your figures will have the same text size and shape as this document (see the \texttt{dummy.py} example within the \texttt{plot\_python} directory).
To check the maximum length (or width) that your figures may have use the following \LaTeX~package \texttt{printlength}| for example the current width is: \verb|\uselengthunit{in}\printlength{\textwidth}| (which returns for this document \uselengthunit{in}\printlength{\textwidth}). There is an example of a Python script and its figure in \cref{fig:dummy}.

When including this type of python figures aspects and other information should not be modified within the \verb|\includegraphics[width=..., height=...]{...}| command, since all that information should have been passed first hand to the python script itself. That way we avoid re-scaling figures, and font matching problems.

\begin{figure}[t]
  \centering
  \includegraphics{dummy.pdf}
  \caption[My first Python figure]{This is a figure made with the script in \texttt{plot\_python/dummy.py}, the figure has the same font format as this caption and text in the document. Notice that $x$-axis and $y$-axis have been written in math mode within the figure labels.}
  \label{fig:dummy}
\end{figure}

To have a well organized code and manuscript I strongly suggest to generate a single figure within a single Python script. Further, call the label similarly as the figure that you have created, so you can then \verb|\ref{fig:dummy}| and know exactly which script created it. Figures, similarly as your thesis will be re-visited multiple times, and people may ask you to modify things on them, so please keep them organized and easy to improve and fix.

\subsection{Ti\textit{k}Z figures}

Ti\textit{k}Z  (PGF plots) is a powerful tool to create in-line-figures in \LaTeX. Unfortunately the learning curve is \textit{very steep}, but it is definitely worth learning it! There are several online examples that you can browse and edit. Making your own figures will always have an extra value in your work and it is highly appreciated. Sources to learn how to make figures are:
\begin{itemize}
    \item The Ti\textit{k}Z package manual, it is long and sometimes hard to grasp but if you want to master Ti\textit{k}Z is the right place to start! \url{https://tikz.dev}
    \item \url{https://www.ctan.org/pkg/pgf}
    \item Several cool examples: \url{http://www.texample.net/}.
\end{itemize}
With some command lines you can make something like in \cref{fig:my_tikz_fig}. Go checkout the code within the \texttt{chapter1/chapter1.tex} directory.

\begin{figure}[t]
  \centering
  \begin{tikzpicture}

    \def\linethick{0.8pt}

    \coordinate (M1) at (-2, 0);
    \coordinate (m) at (2, 4);
    \coordinate (M2) at (4, 0);

    \draw[-latex] (-3, 0) -- (5, 0) node[left, below] {$x$};
    \draw[-latex] (0, -1) -- (0, 5) node[right] {$y$};

    \draw[line width=\linethick] (0, 0) -- node[midway,left] {$r$} (m) node[above] {$m$};
    \draw[line width=\linethick] (M1) node[above, xshift=-10] {$M_1$} -- node[midway,left] {$s_1$} (m);
    \draw[line width=\linethick] (M2) node[above, xshift=7] {$M_2$} -- node[midway,left] {$s_2$} (m);

    \path (M1) -- node[midway, below] {$r_1$} (0, 0);
    \path (0, 0) -- node[midway, below] {$r_2$} (M2);

    % Dashed lines
    \draw[dashed] (M1) -- ++ (0, -1);
    \draw[dashed] (M2) -- ++ (0, -1);

    \draw[|<->|] ($(M1) + (0, -1.1)$) -- node[midway,fill=white] {$a$} ($(M2) + (0, -1.1)$);

    \node[below] at (0.4, 0) {CM};

    \draw ([shift={(0, 0)}]64:0.5) arc[radius=0.5, start angle=64, end angle=0] node[above, xshift=4] {$\theta$};

    % black circles
    \filldraw[black] (m) circle(0.1);
    \filldraw[black] (M1) circle(0.17);
    \filldraw[black] (M2) circle(0.13);
    \filldraw[black] (0, 0) circle(0.05);

  \end{tikzpicture}
  \caption[My first Ti\textit{k}Z figure]{Nice diagram using Ti\textit{k}Z. Notice that the typesetting of variables is the same as the one used in \LaTeX. Variables $\theta$, $M_1$, $M_2$, have the same font and text style. In this caption you should explain all variables shown in the figure.}
  \label{fig:my_tikz_fig}
\end{figure}


\section{Tables}

To make nice tables I suggest the package \texttt{booktabs}, as shown in \cref{tab:my_table}. This lets you add bar lines and make the tables more compact. Since it is a matter of taste feel free to look for your preferred one.

\begin{table}[t]
  \centering
  \caption[My first table]{This is a simple table generated with \texttt{booktabs}. Captions for tables should always go at the top. Similarly as figures we should be using the configuration \texttt{[t]} and let \LaTeX\ automatically sort their position in the text.}
  \label{tab:my_table}
  \begin{tabular}{ccc}
    \toprule
    \textbf{A} & \textbf{B} & \textbf{C} \\ \midrule
    1 & 2 & 3 \\
    1 & 2 & 3 \\
    1 & 2 & 3 \\
    1 & 2 & 3 \\
    1 & 2 & 3 \\
    \bottomrule
  \end{tabular}
\end{table}

\section{Citation and references}

Citations in natural sciences are done in the package \texttt{natbib}. These citation style is the normal in journals such as the Astronomy \& Astrophysics (\url{https://www.aanda.org/component/content?view=article&id=160}), and the Astrophysical Journal. We will not be using the IEEE standard here, but it can be easily implemented. In general, I'd rather using the natural science way since it let us flow the text easily, check the author right away within the text and not to go back and forth over the citations.

YOu should add the \texttt{bibtex} cites in the \texttt{bibfile\_thesis.bib}, then call them within your \LaTeX\ file. Finally, after you add a reference you have to compile the \texttt{pdflatex} once, then compile \texttt{bibtex} and compile pdflatex twice, if not you won't see any changes in your pdf file.

Some in-line example citations:
\begin{itemize}
  \item We have found a fast radio burst to an edge-on galaxy \citep{2024NatAs...8.1429C}.
  \item Authors in \citet{2024NatAs...8.1429C} found a fast radio bursts coming from an edge-on galaxy.
  \item The convolution theorem is one of the most important concepts in radio interferometry \citep[see][chapter~3]{1965ftia.book.....B}.
  \item A new method, developed from first principles, has been developed in very-long baseline interferometry (some authors include \citealt{1965ftia.book.....B}).
\end{itemize}

Please mind double nested parenthesis (in equations figures or citations). To avoid this on citations use the command \verb|\citealt{...}| which will remove the parenthesis. Besides the classical commands \verb|\citep{...}| and \verb|\citet{...}| you can mention the author or year of publication using the commands: \verb|\citeauthor{...}| and \verb|\citeyear{...}|.

Once you have added the citation in the \texttt{bibfile\_thesis.bib} this will not appear in the reference section, unless it is called with the \texttt{citep\{...\}} command. References won't show up when you compile chapters individually, they will only appear once you compile the \texttt{thesis\_msc.tex} file.

To find articles and references it is strongly adviced that you use Astrophysics Data System (ADS): \url{https://ui.adsabs.harvard.edu} here you can find all available references worldwide and easily extract the \texttt{bibtex} string.

\section{Glossaries and acronyms}

The \verb|\usepackage{glossaries}| is vast and large package containing everything to develop and large index, acronyms and other very nice features that may be overkill for an undergraduate thesis document. Nevertheless, it has a simple smaller version to include acronyms. In the \LaTeX\ world there are several other ways to include acronyms, but I think this is one of the best ones. 

To call a command you should define it first in the \texttt{main/main-acronyms.tex} file, and within the text you use the command \verb|\gls{...}|, for example:

\Glspl{frb} are millisecond duration flashes of coherent radio emission. \glspl{frb} appear at cosmological distances and they interact with matter across the way. A single \gls{frb} is capable of outputting the Sun's entire energy in a single burst.

Within the text and this example I used the commands \verb|\Glspl{...}| for capitalized and plural, then \verb|\glspl{...}| for plurals and finally \verb|\gls{...}| for the ordinary case (please check this in the \texttt{chapter1.tex} file).
Lastly, once you have modified the \texttt{main/main-acronyms.tex} file and the thesis itself you have to compile the glossaries with the command: \texttt{makeglossaries main}, in the \texttt{main/} directory.

% To call a glossary \gls{frb}

\section{Boxes}

An additional feature within \LaTeX\ is the capability of defining your own tables or boxes environments. These could very very specific or simple a note. This is, yet another, very large package, \verb|\usepackage{tcolorbox}|. Luckily there's plenty of code out of already built styles and it is as easy as to copy and paste to implement a new table environment.

\begin{note}
  This a simple note. To modify this in your own style go to the \texttt{main.tex} preamble and add the changes that you'd like.
\end{note}